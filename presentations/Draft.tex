% !TEX root = ../Draft.tex
% AEJ-Article.tex for AEA last revised 22 June 2011
\documentclass[AEJ, finalmode]{Formatting/AEA}
\usepackage{Formatting/preamble}

%\documentclass[AEJ]{Formatting/AEA}

% The mathtime package uses a Times font instead of Computer Modern.
% Uncomment the line below if you wish to use the mathtime package:
%\usepackage[cmbold]{mathtime}
% Note that miktex, by default, configures the mathtime package to use commercial fonts
% which you may not have. If you would like to use mathtime but you are seeing error
% messages about missing fonts (mtex.pfb, mtsy.pfb, or rmtmi.pfb) then please see
% the technical support document at http://www.aeaweb.org/templates/technical_support.pdf
% for instructions on fixing this problem.

% Note: you may use either harvard or natbib (but not both) to provide a wider
% variety of citation commands than latex supports natively. See below.

% Uncomment the next line to use the natbib package with bibtex 
\usepackage{natbib}
\usepackage{amsmath,amssymb,amsfonts,url,graphicx}


% define operators used in Methodology
\DeclareMathOperator{\Var}{Var}
\DeclareMathOperator{\Cov}{Cov}
% optional convenience:
\newcommand{\E}{\mathbb{E}}
% Uncomment the next line to use the harvard package with bibtex
%\usepackage[abbr]{harvard}

% This command determines the leading (vertical space between lines) in draft mode
% with 1.5 corresponding to "double" spacing.
%%%%%%%%%%%%%%%%%%%%%%%%%%%%%%%%%%%
%Eye saver mode%
%\usepackage{xcolor}

%\pagecolor[rgb]{0,0,0} %black

%\color[rgb]{0.5,0.5,0.5} %grey
%%%%%%%%%%%%%%%%%%%%%%%%%%%%%%%%%%%%

%%%%%%%%%%%%%%%%%%%%%%%%%%%%%%%%%%


\begin{document}

\title{Right-to-Work, Union Coverage, and the Structure of Pay}
\shortTitle{Right-to-Work, Unions, and Pay}
\author{Alexander Leon-Hernandez}
% \author{Alexander Leon-Hernandez\thanks{%
% Surname1: affiliation1}}
\date{\today}
\pubMonth{December}
\pubYear{2025}
\pubVolume{}
\pubIssue{}
\JEL{}
\Keywords{}

\begin{abstract}
Right-to-work laws weaken unions but their effects on wage levels and inequality are uncertain. 
I estimate how adoptions of right-to-work statutes in five U.S. states from 2003–2019 affected unionization and the wage distribution. 
Using IPUMS-CPS microdata, modern difference-in-differences estimators, and RIF regressions, I find that right-to-work laws reduce union membership and coverage by about 1.5 percentage points but generate modest wage declines, about one percent or less across most of the distribution, with little change in standard inequality measures. 
The results suggest that right-to-work laws mainly lower overall pay rather than reshape the wage structure.
\end{abstract}

\maketitle

% !TEX root = ../Draft.tex
%%[[Preview both the union effects and the wage effects in the intro, with one sentence on magnitudes (e.g., “RTW reduces union coverage by about X p.p. and lowers wages by about Y percent”).
%% One short paragraph on contributions (RTW as policy, modern DiD, RIF distributional stuff).]]
\section*{}

Unions and collective bargaining shape wage setting by raising average pay, compressing wage differences among covered workers, and affecting wages in the broader labor market \citep{FreemanMedoff1984,CardDavid2001,DiNardoJohn1996,CardDavid2003,WesternBruce2011,Farber2018}. 
Cross‑sectional and panel evidence documents a sizable union wage premium and less dispersion of wages within the union sector relative to comparable nonunion jobs \citep{FreemanMedoff1984,BlanchflowerDavidG2003,CardDavid2003}. 
At the same time, changes in union density and bargaining coverage are closely linked to changes in aggregate wage inequality, both over recent decades and over the longer twentieth century \citep{DiNardoJohn1996,WesternBruce2011,Farber2018}. 
This combination of a wage premium, within‑union compression, and spillovers to nonunion wages motivates intense policy and political debate over institutions that weaken unions. 
A central example in the United States is right‑to‑work (RTW) legislation, which prohibits mandatory dues for workers covered by a collective agreement and is widely understood to erode union finances and organizing capacity \citep{Farber2018,EllwoodDavidT1987,IchniowskiCasey1991,FortinNicole2022}. 


This paper studies how the adoption of RTW laws affects the level and distribution of wages in the United States. 
Using IPUMS‑CPS microdata from 2003–2019, we treat RTW as the policy of interest and estimate its reduced‑form impact on wages through all channels, including de‑unionization and weaker union “threat effects.” 
Building on recent work that documents strong links between unionization, spillover effects, and the wage distribution \citep{FortinNicoleM2021,Farber1984}, we exploit the staggered adoption of RTW laws in a modern difference‑in‑differences framework that is robust to heterogeneous treatment effects, and combine it with recentered influence function regressions to recover unconditional effects across the wage distribution.
To capture distributional consequences, we combine these policy‑induced changes with recentered influence function (RIF) regressions, which recover unconditional effects on wage quantiles and related distributional statistics \citep{FirpoSergio2009}.

\subsection{Institutional Background}

Right-to-work (RTW) laws regulate the use of union‐security agreements in the private sector. 
Under the National Labor Relations Act, a union chosen by a majority of workers becomes the exclusive bargaining representative for all workers in the bargaining unit, including nonmembers.  
Section 14(b) of the Taft–Hartley Act permits states to prohibit union‐security clauses that require covered workers to pay union dues or agency fees as a condition of employment.  
In RTW states, workers in unionized workplaces may receive representation and contract benefits without making any financial contribution to the union.\footnote{See, for example, the statutory discussion in \citep{Moore1998} and the overview in the \citep{CongressRTW}.}

Economic and legal research argues that RTW provisions weaken unions through several channels.  
By banning mandatory dues and fees, RTW laws create a free‑rider problem that reduces union revenue per covered worker and raises the cost of organizing and contract enforcement \citep{Farber1984,EllwoodDavidT1987}.  
Empirical studies find that RTW adoption is associated with sizable declines in union density and bargaining coverage in both the private and public sectors \citep{Farber1984,EllwoodDavidT1987,IchniowskiCasey1991,Moore1998}.  
More recent work confirms that RTW laws contribute to the long‑run erosion of unionization in the United States and interact with broader institutional and political trends \citep{Farber2018}.

During our sample period, 2003–2019, five previously non‑RTW states adopted RTW statutes.  
Indiana enacted its law in 2012, Michigan passed a private‑sector RTW statute later in 2012 (effective in 2013), Wisconsin adopted RTW in 2015, West Virginia followed in 2016, and Kentucky implemented RTW legislation in 2017.\footnote{See Indiana Department of Labor (2012); Michigan Public Act 349 of 2012; 2015 Wisconsin Act 1; West Virginia Code §21‑5G‑7; and Kentucky House Bill 1 (2017).\nocite{IN_DOL_RTW_2012,MI_PA_349_2012,WI_Act1_2015,WV_Code_21_5G_7,KY_HB1_2017}}  

\begin{figure}[ht]
  \centering
  \includegraphics[width=\linewidth]{Figures/RTW_map.pdf}
  \caption{Diffusion of Right‑to‑Work Laws Across U.S. States,}
  \label{fig:RTW_map}
  \begin{figurenotes}
States are shaded by the year in which a statewide right-to-work (RTW) law first became effective. Darker shading indicates more recent adoption (2000–2019); white states never adopted RTW through 2019. Numbers within states report the year of first adoption. Indiana, Michigan, Wisconsin, West Virginia, and Kentucky adopted RTW between 2012 and 2017 \citep{IN_DOL_RTW_2012,MI_PA_349_2012,WI_Act1_2015,WV_Code_21_5G_7,KY_HB1_2017}.
\end{figurenotes}
\end{figure}

Figure~\ref{fig:RTW_map} plots the geographic diffusion of RTW laws and highlights these recent adoptions.  
This staggered timing, combined with a large group of never‑RTW comparison states, underlies the difference‑in‑differences design used in the rest of the paper.

\subsection{Related Literature}

This paper builds on several strands of research on unions, wage determination, and RTW legislation.  
A large literature documents that union contracts raise average wages, compress wage dispersion among covered workers, and generate spillovers to wages in the surrounding labor market \citep{FreemanMedoff1984,CardDavid2001,CardDavid2003,DiNardoJohn1996,WesternBruce2011,Farber2018}.  
These studies show that union wage policies flatten wage profiles by skill and seniority and that changes in union density are closely linked to changes in aggregate wage inequality over the postwar period \citep{DiNardoJohn1996,WesternBruce2011,Farber2018}.  

A second literature studies the determinants and consequences of RTW laws.  
Early work exploits cross‑state variation to estimate effects on union membership and coverage, generally finding sizable negative impacts on unionization but more mixed effects on wages and employment \citep{Farber1984,EllwoodDavidT1987,IchniowskiCasey1991,Moore1998}.  
More recent studies use quasi‑experimental variation from newly adopting states.  
\citet{Reed2003} and \citet{MooreNewman1985} find modest wage reductions in RTW states after controlling for industry and worker characteristics, while \citet{ErenOzbeklik2016} use a synthetic control approach for Oklahoma and report limited wage effects but some changes in employment composition.  
Recent empirical work on RTW laws finds that adoption is followed by large and persistent declines in union density, whereas estimated wage and employment effects are generally modest relative to broader macroeconomic forces \citep{Farber2018,ErenOzbeklik2016,FortinNicole2022}.

Methodologically, the paper is related to recent advances in difference‑in‑differences designs with staggered treatment timing.  
Traditional two‑way fixed‑effects estimators can produce biased and hard‑to‑interpret estimates when treatment effects are heterogeneous across cohorts or over time \citep{GoodmanBacon2021}.  
We instead use the estimator of \citet{CallawaySantAnna2021} and the interaction‑weighted approach of \citet{SunAbraham2021} to recover cohort‑specific and dynamic treatment effects that are robust to such heterogeneity.  
To study distributional consequences, we combine these estimators with the recentered influence function (RIF) regression framework of \citet{FirpoSergio2009}, which allows me to interpret RTW effects on unconditional wage quantiles and inequality indices as policy‑induced shifts in the wage distribution.

Relative to the existing RTW literature, this paper makes three contributions.  
First, it treats RTW as the policy of interest and estimates its total reduced‑form effect on wages, rather than using RTW exclusively as an instrument for union status.  
Second, it applies modern difference‑in‑differences estimators to a long panel of IPUMS‑CPS microdata covering the full wave of RTW adoptions since 2012, which provides cleaner identification of dynamic effects than earlier two‑way fixed‑effects designs.  
Third, by combining these methods with RIF regressions, the paper offers a detailed description of how RTW laws shift the entire wage distribution, clarifying whether any wage losses are concentrated in particular parts of the distribution or are broadly shared.



% !TEX root = ../Draft.tex
\section{Data}\label{sec:data}

I use microdata from the Integrated Public Use Microdata Series Current Population Survey (IPUMS CPS; \citet{ipumsCPS2025}). 
IPUMS CPS harmonizes the monthly CPS over time and provides consistent coding and documentation for key labor‑market variables, which is crucial for pooling many years of data and constructing comparable measures of wages, union status, and worker characteristics. 


Our extract consists of the basic monthly CPS from January 2003 through December 2019. 
The starting year is chosen for three reasons. 
First, 2003 marks the introduction of a consistent post‑2000 CPS industry and occupation coding scheme, so beginning the sample then avoids having to splice together different coding systems when defining high‑union industries and occupation groups.\footnote{In earlier years the CPS uses different industry and occupation classifications. Bridging those schemes would introduce additional measurement error in industry‑level unionization and occupation controls without adding new RTW adoptions in our identification period.} 
Second, the main wave of new RTW laws in our period occurs between 2012 and 2017. 
Starting in 2003 provides a long pre‑treatment window (up to nine years) for the earliest adopters while keeping a relatively homogeneous institutional and data environment. 
Third, I end the sample in 2019 to keep the analysis in the pre‑pandemic labor market, avoiding the large and uneven labor‑market disruptions of 2020–2021 that are orthogonal to RTW but would complicate interpretation of dynamic effects. 
Oklahoma adopted RTW in 2001, just before our sample begins, because exploiting that adoption would require extending the CPS series back across earlier coding regimes for industry and occupation, I treat Oklahoma as an always‑RTW state in our analysis.

From each month I keep individuals aged 18–64, employed in the survey week, and classified as private or public wage and salary workers in the ORG earnings universe. 
The extract includes the standard CPS identifiers and weights, state of residence, basic demographics (age, sex, race, Hispanic origin, marital status, education), and labor‑market information (employment status, hours, class of worker, industry and occupation, union status, and hourly wages). 
All statistics and regressions use the appropriate IPUMS CPS person or earnings weights.
The unit of observation in our dataset is an individual worker in a given calendar year. 
I pool the monthly CPS ORG samples within each year and treat the data as a repeated cross-section of worker–year observations; I do not attempt to follow individuals over time. 
Right-to-work status varies at the state–year level and is merged to each worker based on state of residence and survey year.

Union status comes from the ORG union question. 
I classify workers as union members, non‑members covered by a union contract (“free riders”), and not covered by any union. 
From this I form indicators for union membership and for union coverage (member or covered). 
To capture long‑run differences in unionization across industries, I compute industry‑level union coverage using the early part of the sample and define a high‑union‑industry dummy for industries above the median coverage rate.

Right‑to‑work (RTW) status is obtained by merging a state–month panel of RTW laws to the CPS microdata. 
I code a monthly RTW indicator equal to one when a law is in force, and then define a state–year indicator that equals one if RTW is in effect for at least six months of the calendar year. 
From this panel I identify the year in which each state first becomes RTW and use it to classify states into never‑RTW, always‑RTW, and “new adopters’’; for the latter group I distinguish pre‑ and post‑adoption periods in the analysis.

Education is measured using the detailed IPUMS education codes and grouped into four categories: high school or less, some college but no bachelors, bachelors, and graduate or professional degree. 
I map these categories into years of schooling and construct potential experience as age minus schooling minus six, truncating at zero, and include a quadratic in potential experience to be used as part of a Mincer earnings equation. 
Occupations are grouped into eight broad categories (management/business/finance, professional, service, sales, office and administrative support, construction and extraction, production, and transportation and material moving), plus an “unknown’’ category, and I include the corresponding occupation dummies. 
I also construct an indicator for residence in a metropolitan area (MSA) using the CPS \texttt{METRO} and \texttt{METAREA} variables, setting the MSA indicator to one if a respondent is coded as living in a metropolitan or central‑city area (or has a positive metropolitan area code) and zero otherwise, and include a separate dummy for missing MSA information.


\begin{table}[h]
   \caption{Summary Statistics}
    \centering
\vspace{0.3em}\resizebox{\textwidth}{!}{
\begin{tabular}{l d{-3} d{-3} d{-3} d{-3} d{-3} d{-3}}
\toprule
 & \multicolumn{1}{c}{Total} 
 & \multicolumn{1}{c}{Never RTW} 
 & \multicolumn{1}{c}{New RTW (pre)} 
 & \multicolumn{1}{c}{New RTW (post)} 
 & \multicolumn{1}{c}{Always RTW} 
 & \multicolumn{1}{c}{\emph{p}-value} \\
 & \multicolumn{1}{c}{(1)} 
 & \multicolumn{1}{c}{(2)} 
 & \multicolumn{1}{c}{(3)} 
 & \multicolumn{1}{c}{(4)} 
 & \multicolumn{1}{c}{(5)} 
 & \multicolumn{1}{c}{(3)--(2)} \\
\midrule

% ===== UNION STATUS =====
\multicolumn{7}{l}{\textbf{Union status}}\\[0.2em]
\qquad Union covered & 0.131 & 0.174 & 0.160 & 0.128 & 0.070 & 0.000 \\
 & (0.337) & (0.379) & (0.367) & (0.334) & (0.256) & \\[0.2em]
\qquad Union member & 0.118 & 0.162 & 0.149 & 0.117 & 0.057 & 0.000 \\
 & (0.323) & (0.368) & (0.356) & (0.321) & (0.233) & \\[0.2em]

% ===== DEMOGRAPHICS =====
\addlinespace
\multicolumn{7}{l}{\textbf{Demographics}}\\[0.2em]
\qquad Log hourly wage & 2.613 & 2.657 & 2.555 & 2.707 & 2.558 & 0.000 \\
 & (0.505) & (0.510) & (0.494) & (0.489) & (0.496) & \\[0.2em]
\qquad Age & 39.885 & 39.990 & 39.903 & 40.305 & 39.713 & 0.009 \\
 & (12.455) & (12.455) & (12.329) & (12.886) & (12.436) & \\[0.2em]
\qquad Female & 0.482 & 0.483 & 0.488 & 0.485 & 0.479 & 0.000 \\
 & (0.500) & (0.500) & (0.500) & (0.500) & (0.500) & \\[0.2em]

% ===== RACE / ETHNICITY =====
\addlinespace
\multicolumn{7}{l}{\textbf{Race / ethnicity}}\\[0.2em]
\qquad Non-Hispanic White & 0.654 & 0.648 & 0.856 & 0.809 & 0.620 & 0.000 \\
 & (0.476) & (0.478) & (0.351) & (0.393) & (0.485) & \\[0.2em]
\qquad Non-Hispanic Black & 0.121 & 0.099 & 0.076 & 0.093 & 0.159 & 0.000 \\
 & (0.326) & (0.298) & (0.266) & (0.290) & (0.366) & \\[0.2em]
\qquad Hispanic & 0.158 & 0.166 & 0.037 & 0.057 & 0.173 & 0.000 \\
 & (0.364) & (0.372) & (0.189) & (0.232) & (0.378) & \\[0.2em]

% ===== EDUCATION =====
\addlinespace
\multicolumn{7}{l}{\textbf{Education}}\\[0.2em]
\qquad High school or less & 0.365 & 0.348 & 0.397 & 0.349 & 0.384 & 0.000 \\
 & (0.482) & (0.476) & (0.489) & (0.477) & (0.486) & \\[0.2em]
\qquad Some college or Associate of Arts & 0.298 & 0.287 & 0.323 & 0.323 & 0.307 & 0.000 \\
 & (0.457) & (0.452) & (0.467) & (0.468) & (0.461) & \\[0.2em]
\qquad Bachelors or more & 0.336 & 0.365 & 0.280 & 0.327 & 0.309 & 0.000 \\
 & (0.472) & (0.481) & (0.449) & (0.469) & (0.462) & \\[0.2em]

% ===== OCCUPATION / SECTOR =====
\addlinespace
\multicolumn{7}{l}{\textbf{Occupation and sector}}\\[0.2em]
\qquad Professional occupation & 0.141 & 0.149 & 0.121 & 0.134 & 0.135 & 0.000 \\
 & (0.349) & (0.356) & (0.326) & (0.341) & (0.342) & \\[0.2em]
\qquad Service occupation & 0.227 & 0.238 & 0.213 & 0.228 & 0.216 & 0.000 \\
 & (0.419) & (0.426) & (0.409) & (0.419) & (0.412) & \\[0.2em]
\qquad Production occupation & 0.051 & 0.047 & 0.046 & 0.041 & 0.058 & 0.073 \\
 & (0.220) & (0.211) & (0.209) & (0.199) & (0.235) & \\[0.2em]
\qquad Public sector & 0.028 & 0.027 & 0.020 & 0.017 & 0.030 & 0.000 \\
 & (0.164) & (0.163) & (0.139) & (0.129) & (0.171) & \\[0.2em]

\midrule
Number of observations 
  & \multicolumn{1}{c}{2942287} 
  & \multicolumn{1}{c}{1553173} 
  & \multicolumn{1}{c}{175589} 
  & \multicolumn{1}{c}{75809} 
  & \multicolumn{1}{c}{1137716} 
  & \\
\bottomrule
\end{tabular}

    \label{table: Sum Stats RTW}
}
\begin{tablenotes}
    This table reports weighted means and standard deviations (in parentheses) of the listed variables for the full sample  and separately by right-to-work (RTW) status: Never RTW, New RTW (pre), New RTW (post), and Always RTW. 
    All statistics are computed using earnings weights. 
    For each variable, the \emph{p}-value in column (6) comes from a weighted regression of the variable on an indicator for “New RTW (pre)” using only observations in the “Never RTW” and “New RTW (pre)” groups; it thus tests equality of means between columns (2) and (3). 
    The number of observations in each column is shown in the last row. 
\end{tablenotes}
\end{table}



% !TEX root = ../Draft.tex
\section{Empirical Strategy}\label{sec:method}

\subsection{Setup and notation}

The unit of observation is an individual CPS ORG worker in a given survey month. 
We pool monthly IPUMS CPS ORG samples from 2003–2019 and treat the data as a repeated cross-section. 
Let $Y_{ist}$ denote the outcome for worker $i$ living in state $s$ in calendar year $t$. 
In different specifications, $Y_{ist}$ is union coverage, union membership, a free-rider indicator, or log hourly wages. 
Right-to-work status varies at the state–year level. 
Let $D_{st}$ be an indicator equal to one if state $s$ has a RTW law in force in year $t$ (defined below), and zero otherwise. For states that adopt RTW during the sample period, let $G_s$ denote the first year in which $D_{st}=1$; for never-adopting states we set $G_s=0$. 
Our main estimands are the cohort- and event-time specific average treatment effects on the treated, $\text{ATT}(g,t)$, and averages of these effects over post-adoption periods.

\subsection{Treatment timing and groups}

We construct a state–month panel of RTW status and merge it to the CPS file by state and calendar month. 
The monthly RTW indicator equals one beginning in the month when a state's RTW statute becomes effective. 
For the empirical analysis, we aggregate this information to the annual level: $D_{st}=1$ if RTW is in force for at least six months of year $t$ in state $s$, and $D_{st}=0$ otherwise. 
The first year with $D_{st}=1$ defines the cohort $G_s$ for each adopting state. 
States that have RTW in place before 2003 are treated as RTW throughout our sample window and serve as always-treated; states that never adopt RTW by 2019 are never-treated. 
States that switch from $D_{st}=0$ to $D_{st}=1$ between 2003 and 2019 are “new adopters’’ and provide the variation in treatment timing used by the estimator. 
In the event-study graphs we index time relative to adoption, using a symmetric window of five leads and five lags around the year before adoption (event time $-1$) as the omitted baseline.
%% More about different windows in the appendix section

\subsection{Difference-in-differences design}

Identification relies on the standard parallel-trends assumption for staggered adoption: for each cohort $g$, the evolution of untreated potential outcomes for workers in adopting states would have followed that of the comparison states in the absence of RTW. 
We implement the Callaway and Sant'Anna (2021) difference-in-differences estimator for multiple periods and variation in treatment timing. 
Time is indexed by calendar year, the cohort variable is $G_s$, and we use the doubly-robust inverse-probability weighting estimator (\texttt{method(dripw)}). 
Observations are weighted by the CPS earnings weight and standard errors are clustered at the state level. 
For each outcome $Y_{ist}$ we report (i) event-study coefficients, which plot $\text{ATT}(g,t)$ aggregated across cohorts by event time, and (ii) average post-adoption effects that aggregate the cohort- and time-specific ATTs over post-treatment years.

\subsection{Baseline specification and controls}
In all specifications we condition on a common set of worker characteristics. 
Let $X_{ist}$ denote a vector of controls including a quadratic in potential experience (constructed from age and education), dummies for four education groups, indicators for female, race and Hispanic origin, eight broad occupation groups (plus an “unknown’’ category), public-sector employment, and metropolitan residence. 
These controls absorb systematic differences in wages and union status across observable worker types and help stabilize the estimation of propensity scores in the doubly robust procedure. 
We use the same control set for all outcomes (union coverage, union membership, free riding, and log hourly wages), and we always apply the CPS earnings weight and cluster standard errors at the state level, as described above.


Our object of interest is the cohort- and period-specific average treatment effect on the treated (ATT) defined by \citep{CallawaySantAnna2021}. 
Let $Y_{ist}$ denote the outcome for worker $i$ in state $s$ and calendar year $t$, and let $G_s$ denote the first year in which state $s$ adopts a RTW law ($G_s=0$ for never-treated states). 
For a given adopting cohort $g$ and year $t \ge g$, the ATT is
\begin{equation}
\label{eq:att}
\text{ATT}(g,t)
= \mathbb{E}\big[ Y_{ist}(1) - Y_{ist}(0) \,\big|\,
G_s = g,\; t \ge g \big],
\end{equation}
where $Y_{ist}(1)$ and $Y_{ist}(0)$ are potential outcomes with and without RTW. 
Under the parallel-trends and overlap assumptions in \citet{CallawaySantAnna2021}, we can identify $\text{ATT}(g,t)$ by comparing workers in states that adopt RTW in year $g$ to workers in states that are not yet treated (plus never-treated states) in the same year.

We estimate \eqref{eq:att} using the doubly robust difference-in-differences estimator implemented in the \texttt{csdid} command, with calendar year as the time index and $G_s$ as the treatment cohort. 
The estimator combines propensity-score weighting and outcome regression to construct a counterfactual for each treated cohort-year and then recovers $\widehat{\text{ATT}}(g,t)$ for all $g$ and $t$. 
To summarize the dynamics, we aggregate these cohort-specific effects by event time $k=t-g$ and report
\begin{equation}
\label{eq:event}
\delta_k
= \sum_{g} \omega_{g,k} \, \text{ATT}(g, g+k),
\end{equation}
where $\omega_{g,k}$ are nonnegative weights that sum to one over cohorts with $t=g+k$ in the sample. 
The coefficients $\delta_k$ form the event-study profiles plotted in our figures, with $k=-1$ (the year before adoption) normalized to zero.

The vector of worker characteristics $X_{ist}$ that enters the doubly robust adjustment is the same in all specifications cluster standard errors at the state level.


\subsection{Alternative estimator and robustness}
Our baseline estimates rely on the group–time difference‑in‑differences framework of \citet{CallawaySantAnna2021}, implemented with the doubly robust estimator. 
This approach is convenient in our setting because it delivers cohort–time average treatment effects that can be aggregated into overall post‑adoption effects and used consistently across the various outcomes we study, including RIF‑based measures of wage quantiles and subsamples defined by industry unionization. 
As a robustness check, we also estimate dynamic effects using the interaction‑weighted event‑study estimator of \citet{SunAbraham2021}, which likewise corrects the weighting problems of two‑way fixed‑effects models with staggered adoption. 

Appendix Figure A shows that the Sun–Abraham event‑study profiles are very similar to our baseline csdid estimates for unionization and wages, so we focus on the Callaway–Sant’Anna results in the main text.

%[[test]] 
\subsection{Distributional effects via RIF regressions}

Beyond average wage effects, we also study how RTW laws affect different parts of the wage distribution. 
Following \citet{FirpoSergio2009}, we use recentered influence function (RIF) regressions for unconditional quantiles of log hourly wages. For a given unconditional quantile $\tau$ of $\ln(\text{wage})$, we construct the RIF of log wages at $\tau$ and treat this transformed variable as the outcome in the same difference-in-differences design described above. 
The resulting coefficients can be interpreted as changes in the unconditional $\tau$th quantile of log wages.

Operationally, we estimate separate Callaway--Sant'Anna DID specifications with the RIF of log hourly wages as the dependent variable at several quantiles (e.g.\ the 10th, 25th, 50th, 75th, and 90th unconditional quantiles). 
We keep the same sample, weighting scheme, and set of controls as in the baseline mean-wage specification, and continue to cluster standard errors by state. 
In the results section, we summarize these estimates by plotting the ATT on log wages across quantiles, which shows how RTW affects the wage distribution rather than only the mean.%

\subsection{Heterogeneity by industry unionization}

We also examine whether RTW effects differ between more and less unionized industries. Using the pre-period CPS data, we compute industry-level union coverage rates (weighted by earnings weights) and define a high-union-industry indicator equal to one for industries with coverage above the median and zero otherwise. This industry measure is merged back to the full worker sample. To study heterogeneity, we re-estimate the baseline CSDID specifications separately for workers in high- and low-union industries, focusing on union outcomes and log wages. Comparing the resulting event-study profiles and average effects across these two subsamples allows us to assess whether RTW laws have stronger impacts in segments of the labor market where unions were initially more prevalent.


\section{Empirical Results} \label{sec:results}

This section presents the effects of right-to-work (RTW) adoption on unionization and wages. 
I first document the dynamic response of union membership and coverage, then turn to mean wage effects, distributional effects across the wage distribution, and heterogeneity by industry unionization.

\subsection{Effects on union outcomes}

Figure~2 plots event-study estimates of the impact of RTW laws on union membership and coverage. 
The pre-treatment coefficients are small and statistically indistinguishable from zero, indicating no strong differential trends in union outcomes between new-RTW states and the comparison group in the years leading up to adoption. 
After RTW takes effect, both membership and coverage decline steadily over the subsequent four to five years, with the point estimates stabilizing around a loss of roughly 1.5–2.5 percentage points relative to the pre-adoption baseline.

Table~\ref{tab:rtw_union_wage} summarizes these patterns by reporting average post-adoption treatment effects. 
In the baseline specification with controls (column 2), RTW adoption reduces the union membership rate by 1.5 percentage points and the coverage rate by a similar 1.5 percentage points. 
Given pre-adoption unionization rates in new-RTW states, these estimates correspond to roughly a ten percent decline in both membership and coverage. 
By contrast, the estimated effect on the free-rider rate is essentially zero, suggesting that RTW operates mainly by shrinking the share of workers in unionized workplaces rather than by inducing large changes in the member–nonmember composition within covered workplaces.

Columns (3) and (4) of Table~\ref{tab:rtw_union_wage} show that these unionization effects are highly uneven across the labor market. 
In industries that were more unionized in the pre-period, RTW adoption is followed by a decline of about 2.6 percentage points in union membership and 2.4 percentage points in coverage, whereas the corresponding effects in less unionized industries are near zero. 
This pattern is consistent with the view that RTW laws bite hardest in segments of the labor market where unions were initially more entrenched.

\begin{figure}[ht]
    \centering
    \caption{Impact of RTW on Union Outcomes}
    \captionsetup[subfigure]{labelformat=empty}
 \begin{subfigure}{0.49\textwidth}
\includegraphics[width=\linewidth]{Figures/csdid/Full_mem.pdf}
  \caption{Panel A. Union membership rate}
    \label{fig:Full_mem}
 \end{subfigure}
    \hfill
 \begin{subfigure}{0.49\textwidth}
\includegraphics[width=\linewidth]{Figures/csdid/Full_cov.pdf}
  \caption{Panel B. Union coverage rate}
    \label{fig:Full_cov}
 \end{subfigure}
    \begin{figurenotes}
        This figure plots event-study estimates of the impact of right-to-work (RTW) laws on union outcomes using the \citep{CallawaySantAnna2021} difference-in-differences estimator. 
        The horizontal axis measures years relative to a state's first RTW adoption ($t=0$), and the vertical axis reports average treatment effects on the treated (ATT). 
        Coefficients are estimated for leads and lags from $t=-5$ to $t=+5$ and normalized so that the coefficient in $t=-1$ is zero. 
        Points show estimated ATTs, vertical bars show 95 percent confidence intervals. 
        The vertical line at $t=0$ marks the year of RTW adoption. 
        Panel A uses an indicator for union membership as the outcome; Panel B uses an indicator for union coverage (including nonmember workers covered by a union contract). 
        All specifications control for potential experience (and its square), education, gender, race/ethnicity, occupation, public-sector status, and MSA fixed effects, are weighted by earnings weights, and cluster standard errors at the state level.
    \end{figurenotes}
\end{figure}

\begin{table}[htbp]
    \centering
    \caption{Baseline right-to-work effects on union outcomes and log hourly wages}
    \label{tab:rtw_union_wage}
\vspace{0.3em}\resizebox{\textwidth}{!}{
\begin{tabular}{l d{-3} d{-3} d{-3} d{-3}}
\toprule
 & \multicolumn{1}{c}{No controls} & \multicolumn{1}{c}{Baseline csdid} & \multicolumn{1}{c}{High-union states} & \multicolumn{1}{c}{Low-union states} \\
 & \multicolumn{1}{c}{(1)} & \multicolumn{1}{c}{(2)} & \multicolumn{1}{c}{(3)} & \multicolumn{1}{c}{(4)} \\
\midrule
Union membership rate & -0.015 & -0.015 & -0.026 & -0.000 \\\\
 & ( 0.003) & ( 0.004) & ( 0.006) & ( 0.002) \\\\[0.3em]
Union coverage rate & -0.014 & -0.015 & -0.024 & -0.002 \\\\
 & ( 0.003) & ( 0.003) & ( 0.006) & ( 0.002) \\\\[0.3em]
Free-rider rate &  0.000 &  0.000 &  0.002 & -0.002 \\\\
 & ( 0.001) & ( 0.001) & ( 0.001) & ( 0.001) \\\\[0.3em]
Log hourly wage & -0.006 & -0.000 & -0.005 &  0.002 \\\\
 & ( 0.009) & ( 0.006) & ( 0.005) & ( 0.008) \\\\[0.3em]
\midrule
Number of observations & \multicolumn{1}{c}{  1052315} & \multicolumn{1}{c}{  1049953} & \multicolumn{1}{c}{   620970} & \multicolumn{1}{c}{   423663} \\\\
\bottomrule
\end{tabular}

}
\begin{tablenotes}
    Each cell reports the average post-adoption treatment effect of right-to-work (RTW) laws on the listed outcome using the \citet{CallawaySantAnna2021} difference-in-differences estimator, implemented with the doubly robust inverse-probability-weighted method. 
    Column (1) uses a specification without worker covariates. 
    Column (2) reports the baseline csdid specification with controls for potential experience (and its square), educational attainment, gender, race/ethnicity, broad occupation group, public-sector status, and metropolitan residence. 
    Columns (3) and (4) re-estimate the baseline specification separately for workers in high- and low-union industries, defined using pre-RTW industry union coverage above or below the median. 
    All regressions drop always-RTW states, apply CPS earnings weights, and cluster standard errors at the state level. 
    Standard errors are reported in parentheses.
\end{tablenotes}
\end{table}

\subsection{Effects on average wages}

Figure~3 turns to log hourly wages. 
The event-study coefficients again show little evidence of systematic pre-trends, wage effects in the five years prior to RTW adoption are close to zero and statistically insignificant. 
After adoption, the point estimates drift slightly downward, with the largest negative effects on the order of one to two log points (about one to two percent), but the confidence intervals are wide and include zero in most post-treatment years.

The baseline csdid estimates in Table~\ref{tab:rtw_union_wage} confirm that average wage effects are modest. 
In the specification with controls (column 2), the estimated average post-adoption effect on log hourly wages is effectively zero, with a standard error of 0.006. 
This implies a 95 percent confidence interval of roughly plus or minus 1.2 percent, so the data are consistent with small wage gains or losses but rule out large average effects of RTW on hourly pay over the sample period. 
Together with the event-study evidence, this suggests that the substantial post-RTW declines in unionization do not translate into commensurately large short- or medium-run changes in mean wages.

\begin{figure}[ht]
    \centering
    \caption{Impact of RTW on log Hourly Wage}
    \includegraphics[width=\linewidth]{Figures/csdid/Full_lnwage.pdf}
    \begin{figurenotes}
    This figure plots event-study estimates of the effects of right-to-work (RTW) laws on log hourly wages using the \citep{CallawaySantAnna2021} difference-in-differences estimator, implemented with the doubly robust inverse-probability-weighted method. 
    The horizontal axis measures years relative to a state's first RTW adoption ($t = 0$), and the vertical axis reports average treatment effects on the treated (ATT) in log wage units. 
    Coefficients are estimated for leads and lags from $t = -5$ to $t = +5$ and normalized so that the coefficient in $t = -1$ is zero. 
    Points show estimated ATTs and vertical bars show 95 percent confidence intervals; the vertical line at $t = 0$ marks the year of RTW adoption.
    The estimation sample drops states that are always under a RTW law over the sample period, applies earnings weights, and controls for potential experience (and its square), educational attainment, gender, race/ethnicity, occupation, public-sector status, and MSA fixed effects; standard errors are clustered at the state level. For small effects, a value of 0.01 on the vertical axis corresponds approximately to a 1 percent change in hourly wages.
    \end{figurenotes}
\end{figure}

\subsection{Distributional effects on wages}

Average effects may mask heterogeneous impacts across the wage distribution. 
Figure~4 summarizes the RIF–csdid estimates for unconditional quantiles of log hourly wages. 
The point estimates are small at all quantiles, generally within a band of plus or minus two percent, and the confidence intervals are wide. 
At the very bottom of the distribution (10th percentile) the point estimate is slightly positive, while at the upper quantiles the estimates are modestly negative, but none of these differences is precisely estimated.

The near-uniformity of the quantile-specific effects implies little change in overall wage dispersion. 
The implied effects on standard inequality measures, such as the 90–10 and 50–10 wage gaps, are therefore small, with confidence intervals that comfortably include zero. 
In other words, within the period and set of states studied here, RTW laws appear to nudge the entire wage distribution down by at most a few percent, rather than sharply widening or narrowing wage inequality.


\begin{figure}[ht]
    \centering
    \caption{}
    \includegraphics[width=\linewidth]{Figures/rtw_rif_quantiles_pctwage_csdid.pdf}
    \begin{figurenotes}
        This figure plots RIF–csdid estimates of the effect of right-to-work (RTW) laws on unconditional quantiles of log hourly wages. 
        The horizontal axis indexes unconditional wage quantiles (10th, 25th, 50th, 75th, and 90th percentiles); the vertical axis reports the implied percent change in hourly wages at each quantile. 
        Points show average post-adoption ATTs and vertical bars show 95 percent confidence intervals. 
        Estimates use the \citet{CallawaySantAnna2021} difference-in-differences estimator with recentered influence functions of log wages as outcomes, the same sample and control set as the baseline wage specification, CPS earnings weights, and standard errors clustered at the state level.
    \end{figurenotes}
\end{figure}

\subsection{Heterogeneity by industry unionization}

Finally, I examine whether RTW effects differ between more and less unionized industries. 
Columns (3) and (4) of Table~\ref{tab:rtw_union_wage} show that the declines in union membership and coverage are concentrated in high-union industries, in those sectors, RTW reduces membership by about 2.6 percentage points and coverage by 2.4 percentage points, whereas in low-union industries the corresponding estimates are close to zero. 
This pattern suggests that RTW mainly erodes union presence where it was initially strongest.

Figure~5 reports event-study estimates of log wages separately for below- and above-median union industries. 
In low-union industries, the wage coefficients fluctuate around zero both before and after adoption, with no clear persistent pattern. 
In contrast, in high-union industries the post-RTW coefficients tend to be somewhat more negative, with point estimates that reach roughly three log points (about three percent) in some post-treatment years, although the associated confidence intervals are wide. 
These results are broadly consistent with somewhat larger wage losses in more union-intensive sectors, but they are imprecisely estimated and remain modest in absolute size.


\begin{figure}[ht]
    \centering
    \caption{Impact of RTW on Log Hourly Wage}
    \captionsetup[subfigure]{labelformat=empty}

    \begin{subfigure}{0.49\textwidth}
        \caption*{Panel A}
        \includegraphics[width=\linewidth]{Figures/csdid/0_UNFull_lnwage.pdf}
        \caption*{Below Median Union Membership Rate}
        \label{fig:Full_mem}
    \end{subfigure}
    \hfill
    \begin{subfigure}{0.49\textwidth}
        \caption*{Panel B}
        \includegraphics[width=\linewidth]{Figures/csdid/1_UNFull_lnwage.pdf}
        \caption*{Above Median Union Membership Rate}
        \label{fig:Full_cov}
    \end{subfigure}

    \begin{figurenotes}
    This figure plots event-study estimates of the effects of right-to-work (RTW) laws on log hourly wages using the \citep{CallawaySantAnna2021} difference-in-differences estimator, implemented with the doubly robust inverse-probability-weighted method. 
    The horizontal axis measures years relative to a state's first RTW adoption ($t = 0$), and the vertical axis reports average treatment effects on the treated (ATT) in log wage units. 
    Coefficients are estimated for leads and lags from $t = -5$ to $t = +5$ and normalized so that the coefficient in $t = -1$ is zero. 
    Points show estimated ATTs and vertical bars show 95 percent confidence intervals; the vertical line at $t = 0$ marks the year of RTW adoption.
    The estimation sample drops states that are always under a RTW law over the sample period, applies earnings weights, and controls for potential experience (and its square), educational attainment, gender, race/ethnicity, occupation, public-sector status, and MSA fixed effects; standard errors are clustered at the state level. For small effects, a value of 0.01 on the vertical axis corresponds approximately to a 1 percent change in hourly wages.
    Additionally the estimation sample in Panel A is restricted to below median unionized industries and similar for Panel B with above median.
    \end{figurenotes}
\end{figure}


\section{Discussion}\label{sec:discussion}

The results paint a clear picture of how RTW laws affect unions and wages in the recent wave of adoptions. 
RTW statutes lead to large and persistent reductions in union membership and coverage, especially in sectors that were initially more unionized. 
By contrast, the estimated effects on average wages are small and imprecise, and the distributional analysis reveals little systematic change in wage inequality. 
Taken together, the evidence suggests that RTW laws primarily shift the overall level of unionization and, at most, nudge the wage distribution downward by a modest amount.

These findings speak to ongoing debates about the role of unions in shaping the wage structure. 
Classic work documents a sizeable union wage premium, within-union wage compression, and spillovers to nonunion wages, and argues that the long-run decline of unions has contributed to rising wage inequality.\footnote{See, among others, \citet{FreemanMedoff1984,DiNardoJohn1996,WesternBruce2011,Farber2018}.} 
The evidence here is more narrowly focused on the marginal effect of recent RTW adoptions in five states between 2012 and 2017, holding fixed the broader institutional environment. 
Within that margin, RTW adoption sharply reduces unionization but does not generate correspondingly large increases in measured wage dispersion, at least over the time horizon captured by the CPS data.

The pattern of results is consistent with several mechanisms. 
First, the fact that RTW lowers both membership and coverage, with little change in free-riding, suggests that the dominant channel is a contraction in the incidence of unionized workplaces rather than a dramatic shift in the composition of workers inside those workplaces. 
Second, unions may already have been too weak in many of the newly adopting states for additional RTW-induced declines to translate into sizeable changes in bargaining outcomes, especially against the backdrop of national forces affecting wages. 
Third, firms and workers may adjust along non-wage margins—such as benefits, hours, or job security—that are not fully captured by the CPS wage measure.

At the same time, the results highlight that RTW can generate modest wage losses that are broadly shared rather than tightly concentrated at the bottom of the distribution. 
The RIF-based estimates show similar, small effects across the 10th, 50th, and 90th percentiles, implying little change in standard inequality measures. 
In this sense, RTW laws in the recent period appear to operate mainly as a level shift to pay rather than as a mechanism that reshapes the wage structure.

Finally, several limitations should temper the interpretation of these findings. 
The identification strategy relies on parallel trends and no-anticipation assumptions, which are supported by the flat pre-treatment event-study coefficients but cannot be tested directly. 
The estimates are also based on a relatively small set of adopting states, so they speak most directly to the marginal effect of RTW expansions in the 2010s rather than to earlier episodes or to potential future adoptions in very different institutional environments. 
Moreover, the analysis focuses on hourly wages and does not capture potential effects on employment, hours, non-wage benefits, or firm-level outcomes. 
These caveats notwithstanding, the evidence indicates that, in recent experience, RTW laws are best characterized as policies that substantially weaken unions while exerting modest and fairly uniform downward pressure on wages.



% !TEX root = ../Draft.tex
\section{Conclusion}
This paper has examined how recent adoptions of right-to-work (RTW) laws in the United States have affected unionization and the structure of pay. 
Using CPS microdata from 2003–2019, modern difference-in-differences estimators that account for staggered treatment timing, and RIF regressions for unconditional wage quantiles, I treated RTW statutes as the policy of interest and estimated their reduced-form impact on union outcomes and the wage distribution.

The results show that RTW laws lead to sizable and persistent declines in unionization. 
In the five states that adopted RTW between 2012 and 2017, union membership and coverage fall by roughly 1.5 percentage points on average, corresponding to about a ten percent decline relative to pre-adoption levels. 
These effects are concentrated in industries that were initially more unionized, where RTW adoption reduces membership and coverage by roughly 2.5 percentage points, while impacts in low-union industries are close to zero. 
By contrast, the free-rider rate changes little, suggesting that the main margin of adjustment is the incidence of unionized workplaces rather than the composition of workers within them.

Despite these sizable declines in unionization, the estimated wage effects are modest. 
Average hourly wages fall by at most about one percent, and the RIF-based estimates indicate small and fairly uniform impacts across the wage distribution. 
The implied changes in standard inequality measures such as the 90–10 and 50–10 gaps are close to zero. 
In this recent wave of adoptions, RTW laws therefore appear to operate primarily by weakening unions and nudging the overall level of pay slightly downward, rather than by sharply reshaping wage inequality.

These findings have two main implications. 
First, they suggest that the marginal effect of RTW in the 2010s operates in a context where unions are already substantially eroded, so further reductions in unionization translate into only limited changes in wage setting, at least in the short to medium run. 
Second, they underscore that RTW can still matter for workers’ earnings, but primarily through a broad, modest level shift in wages rather than through dramatic polarization of the wage distribution.

Several limitations point to directions for future research. 
The estimates rely on parallel-trends and no-anticipation assumptions and are identified from a small set of adopting states in a particular institutional period. 
They also focus on hourly wages and do not consider employment, hours, benefits, or firm-level outcomes, which may respond differently to RTW. 
Extending the analysis to these margins, to longer horizons, and to heterogeneous effects by worker characteristics or bargaining context would help to build a more complete picture of how RTW laws shape labor-market outcomes.
% !TEX root = ../Draft.tex
%\section{Appendix}


\bibliographystyle{aea} 
\bibliography{RTW_Project_Lit_Review}
\end{document}

