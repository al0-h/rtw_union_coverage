% !TEX root = ../Draft.tex
%%[[Preview both the union effects and the wage effects in the intro, with one sentence on magnitudes (e.g., “RTW reduces union coverage by about X p.p. and lowers wages by about Y percent”).
%% One short paragraph on contributions (RTW as policy, modern DiD, RIF distributional stuff).]]
\section*{}

Unions and collective bargaining shape wage setting by raising average pay, compressing wage differences among covered workers, and affecting wages in the broader labor market \citep{FreemanMedoff1984,CardDavid2001,DiNardoJohn1996,CardDavid2003,WesternBruce2011,Farber2018}. 
Cross‑sectional and panel evidence documents a sizable union wage premium and less dispersion of wages within the union sector relative to comparable nonunion jobs \citep{FreemanMedoff1984,BlanchflowerDavidG2003,CardDavid2003}. 
At the same time, changes in union density and bargaining coverage are closely linked to changes in aggregate wage inequality, both over recent decades and over the longer twentieth century \citep{DiNardoJohn1996,WesternBruce2011,Farber2018}. 
This combination of a wage premium, within‑union compression, and spillovers to nonunion wages motivates intense policy and political debate over institutions that weaken unions. 
A central example in the United States is right‑to‑work (RTW) legislation, which prohibits mandatory dues for workers covered by a collective agreement and is widely understood to erode union finances and organizing capacity \citep{Farber2018,EllwoodDavidT1987,IchniowskiCasey1991,FortinNicole2022}. 


This paper studies how the adoption of RTW laws affects the level and distribution of wages in the United States. 
Using IPUMS‑CPS microdata from 2003–2019, we treat RTW as the policy of interest and estimate its reduced‑form impact on wages through all channels, including de‑unionization and weaker union “threat effects.” 
Building on recent work that documents strong links between unionization, spillover effects, and the wage distribution \citep{FortinNicoleM2021,Farber1984}, we exploit the staggered adoption of RTW laws in a modern difference‑in‑differences framework that is robust to heterogeneous treatment effects, and combine it with recentered influence function regressions to recover unconditional effects across the wage distribution.
To capture distributional consequences, we combine these policy‑induced changes with recentered influence function (RIF) regressions, which recover unconditional effects on wage quantiles and related distributional statistics \citep{FirpoSergio2009}.

\subsection{Institutional Background}

Right-to-work (RTW) laws regulate the use of union‐security agreements in the private sector. 
Under the National Labor Relations Act, a union chosen by a majority of workers becomes the exclusive bargaining representative for all workers in the bargaining unit, including nonmembers.  
Section 14(b) of the Taft–Hartley Act permits states to prohibit union‐security clauses that require covered workers to pay union dues or agency fees as a condition of employment.  
In RTW states, workers in unionized workplaces may receive representation and contract benefits without making any financial contribution to the union.\footnote{See, for example, the statutory discussion in \citep{Moore1998} and the overview in the \citep{CongressRTW}.}

Economic and legal research argues that RTW provisions weaken unions through several channels.  
By banning mandatory dues and fees, RTW laws create a free‑rider problem that reduces union revenue per covered worker and raises the cost of organizing and contract enforcement \citep{Farber1984,EllwoodDavidT1987}.  
Empirical studies find that RTW adoption is associated with sizable declines in union density and bargaining coverage in both the private and public sectors \citep{Farber1984,EllwoodDavidT1987,IchniowskiCasey1991,Moore1998}.  
More recent work confirms that RTW laws contribute to the long‑run erosion of unionization in the United States and interact with broader institutional and political trends \citep{Farber2018}.

During our sample period, 2003–2019, five previously non‑RTW states adopted RTW statutes.  
Indiana enacted its law in 2012, Michigan passed a private‑sector RTW statute later in 2012 (effective in 2013), Wisconsin adopted RTW in 2015, West Virginia followed in 2016, and Kentucky implemented RTW legislation in 2017.\footnote{See Indiana Department of Labor (2012); Michigan Public Act 349 of 2012; 2015 Wisconsin Act 1; West Virginia Code §21‑5G‑7; and Kentucky House Bill 1 (2017).\nocite{IN_DOL_RTW_2012,MI_PA_349_2012,WI_Act1_2015,WV_Code_21_5G_7,KY_HB1_2017}}  

\begin{figure}[ht]
  \centering
  \includegraphics[width=\linewidth]{Figures/RTW_map.pdf}
  \caption{Diffusion of Right‑to‑Work Laws Across U.S. States,}
  \label{fig:RTW_map}
  \begin{figurenotes}
States are shaded by the year in which a statewide right-to-work (RTW) law first became effective. Darker shading indicates more recent adoption (2000–2019); white states never adopted RTW through 2019. Numbers within states report the year of first adoption. Indiana, Michigan, Wisconsin, West Virginia, and Kentucky adopted RTW between 2012 and 2017 \citep{IN_DOL_RTW_2012,MI_PA_349_2012,WI_Act1_2015,WV_Code_21_5G_7,KY_HB1_2017}.
\end{figurenotes}
\end{figure}

Figure~\ref{fig:RTW_map} plots the geographic diffusion of RTW laws and highlights these recent adoptions.  
This staggered timing, combined with a large group of never‑RTW comparison states, underlies the difference‑in‑differences design used in the rest of the paper.

\subsection{Related Literature}

This paper builds on several strands of research on unions, wage determination, and RTW legislation.  
A large literature documents that union contracts raise average wages, compress wage dispersion among covered workers, and generate spillovers to wages in the surrounding labor market \citep{FreemanMedoff1984,CardDavid2001,CardDavid2003,DiNardoJohn1996,WesternBruce2011,Farber2018}.  
These studies show that union wage policies flatten wage profiles by skill and seniority and that changes in union density are closely linked to changes in aggregate wage inequality over the postwar period \citep{DiNardoJohn1996,WesternBruce2011,Farber2018}.  

A second literature studies the determinants and consequences of RTW laws.  
Early work exploits cross‑state variation to estimate effects on union membership and coverage, generally finding sizable negative impacts on unionization but more mixed effects on wages and employment \citep{Farber1984,EllwoodDavidT1987,IchniowskiCasey1991,Moore1998}.  
More recent studies use quasi‑experimental variation from newly adopting states.  
\citet{Reed2003} and \citet{MooreNewman1985} find modest wage reductions in RTW states after controlling for industry and worker characteristics, while \citet{ErenOzbeklik2016} use a synthetic control approach for Oklahoma and report limited wage effects but some changes in employment composition.  
Recent empirical work on RTW laws finds that adoption is followed by large and persistent declines in union density, whereas estimated wage and employment effects are generally modest relative to broader macroeconomic forces \citep{Farber2018,ErenOzbeklik2016,FortinNicole2022}.

Methodologically, the paper is related to recent advances in difference‑in‑differences designs with staggered treatment timing.  
Traditional two‑way fixed‑effects estimators can produce biased and hard‑to‑interpret estimates when treatment effects are heterogeneous across cohorts or over time \citep{GoodmanBacon2021}.  
We instead use the estimator of \citet{CallawaySantAnna2021} and the interaction‑weighted approach of \citet{SunAbraham2021} to recover cohort‑specific and dynamic treatment effects that are robust to such heterogeneity.  
To study distributional consequences, we combine these estimators with the recentered influence function (RIF) regression framework of \citet{FirpoSergio2009}, which allows me to interpret RTW effects on unconditional wage quantiles and inequality indices as policy‑induced shifts in the wage distribution.

Relative to the existing RTW literature, this paper makes three contributions.  
First, it treats RTW as the policy of interest and estimates its total reduced‑form effect on wages, rather than using RTW exclusively as an instrument for union status.  
Second, it applies modern difference‑in‑differences estimators to a long panel of IPUMS‑CPS microdata covering the full wave of RTW adoptions since 2012, which provides cleaner identification of dynamic effects than earlier two‑way fixed‑effects designs.  
Third, by combining these methods with RIF regressions, the paper offers a detailed description of how RTW laws shift the entire wage distribution, clarifying whether any wage losses are concentrated in particular parts of the distribution or are broadly shared.

