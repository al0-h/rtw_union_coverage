% !TEX root = ../Draft.tex
\section{Conclusion}
This paper has examined how recent adoptions of right-to-work (RTW) laws in the United States have affected unionization and the structure of pay. 
Using CPS microdata from 2003–2019, modern difference-in-differences estimators that account for staggered treatment timing, and RIF regressions for unconditional wage quantiles, I treated RTW statutes as the policy of interest and estimated their reduced-form impact on union outcomes and the wage distribution.

The results show that RTW laws lead to sizable and persistent declines in unionization. 
In the five states that adopted RTW between 2012 and 2017, union membership and coverage fall by roughly 1.5 percentage points on average, corresponding to about a ten percent decline relative to pre-adoption levels. 
These effects are concentrated in industries that were initially more unionized, where RTW adoption reduces membership and coverage by roughly 2.5 percentage points, while impacts in low-union industries are close to zero. 
By contrast, the free-rider rate changes little, suggesting that the main margin of adjustment is the incidence of unionized workplaces rather than the composition of workers within them.

Despite these sizable declines in unionization, the estimated wage effects are modest. 
Average hourly wages fall by at most about one percent, and the RIF-based estimates indicate small and fairly uniform impacts across the wage distribution. 
The implied changes in standard inequality measures such as the 90–10 and 50–10 gaps are close to zero. 
In this recent wave of adoptions, RTW laws therefore appear to operate primarily by weakening unions and nudging the overall level of pay slightly downward, rather than by sharply reshaping wage inequality.

These findings have two main implications. 
First, they suggest that the marginal effect of RTW in the 2010s operates in a context where unions are already substantially eroded, so further reductions in unionization translate into only limited changes in wage setting, at least in the short to medium run. 
Second, they underscore that RTW can still matter for workers’ earnings, but primarily through a broad, modest level shift in wages rather than through dramatic polarization of the wage distribution.

Several limitations point to directions for future research. 
The estimates rely on parallel-trends and no-anticipation assumptions and are identified from a small set of adopting states in a particular institutional period. 
They also focus on hourly wages and do not consider employment, hours, benefits, or firm-level outcomes, which may respond differently to RTW. 
Extending the analysis to these margins, to longer horizons, and to heterogeneous effects by worker characteristics or bargaining context would help to build a more complete picture of how RTW laws shape labor-market outcomes.